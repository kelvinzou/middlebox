\documentclass[twoside,leqno, 11pt]{article}  
%twocolumn
\usepackage{ltexpprt} 
\usepackage{url}
\usepackage{framed}
\usepackage{amsmath}
\usepackage{authblk}
\usepackage{cite}
\usepackage{verbatim}
\usepackage{graphicx}
\begin{document}
\title{\Large  Proposal: Joint Optimization of Routing and Node Selection}
%\thanks{Supported by .} 
\author[]{X. Kelvin Zou}
\affil[]{Department of Computer Science, Princeton University}
\affil[ ]{\textit {\{ xuanz\}@cs.princeton.edu}}
\date{}

\maketitle

%%%%%%%%%%%%%%%%%%%%%%%%%%%%%%%%%%%%%
%%%%%%%%%%                         Intro                                 %%%%%
%%%%%%%%%%%%%%%%%%%%%%%%%%%%%%%%%%%%%

\section{Motivation}

Middlebox is big organic component in networking ecology. For long time due to the fact that the networking only speaks various protocols and most routing are destination based, the placement for middleboxes are ossified: placing them at choke points. This solution has quite a few drawbacks, people either give up the fault-tolerance and thus prone to link failure and overloading, or purchase over provisioned machines to for redundancy and incurs unnecessary cost. 

Today with the help of Software Defined Networks(SDN) the deployment of middleboxes are much more flexible than ever before. People have explored various options of deploying middleboxes in networks \cite{SIMPLE2013, APLOMB2012,ANANTA2013}. However those approaches usually steer traffic as an overlay on top of the network, e.g. it abstract the links between the source traffic and some middlebox it needs to go through as a big abstract link. and all the congestion, load balancing are handled in substrate network layer.

Though the model sounds promising, the approaches are more focused on feasibility of these functions and architectural redesign, therefore one big piece is missing: how to make joint optimal choice for middlebox placement and routing if you have the networking demands based on either historical data or traffic under control.

 \section{Problem formulation} 
 
 In this paper, we introduce a network flow model. Max flow is well known problem \cite{FordFulkerson, Edmonds1972} and it leads to a lot of interesting research on networking design. If we review the network abstraction, it is essentially composed of nodes, links and flows. However the model is missing one important piece for networking. Network flows are not simply just like water, a lot of processing work are carried via flows and need to be done before the flow reaches destination. What if flow needs to be processed at some node along the path? What is more interesting is that due to the fact that each node has a certain processing capacity, we cannot actually handle the flow at a single node, but instead they are handled at different nodes, saying we can handle half at the first hop and the other half at the second node. This essentially leads to an interesting graph model: we have a graph G(V, E), with both link bandwidth \textbf{B}(e)where $e \in E(u,v)$ and node capacity \textbf{C}(v)where $v \in V $, and we are trying to fill in traffic demand with both traffic rate demand \textbf{R}(src, dest) and computational work demand \textbf{D}(src, dest). \textbf{R}(src, dest) should be correlated to \textbf{D}(src, dest). Throughout the whole paper, we assume linear relation $q_i = \frac{D}{R}$ for each flow $ i_{src,dst} $right now, which covers most cases, since most workload processing is per packet based. The system can achieve a joint optimal outcome for both routing and node workload allocation usage by coupling routing problem to processing resource placement and processing resource allocation problem.  
 
\textbf{Hard Constraints}:
\newline
\textbf{B}(e): bandwidth capacity for link e $\in E_{u,v}$
 \newline
\textbf{C}(v): middlebox processing capacity at node v $\in N_{v}$
 \newline
$ \boldsymbol{R}_{i} \text{: flow rate R for } i_{src, dst}$ 
\newline
$ \boldsymbol{D}_{i} \text{: flow middlebox processing demand for } i_{src, dst}$
\newline 
$ q_{i} \text{: ratio of D to R for flow } i_{src, dst}$ 
%%%%%%%%%%%%%%%%%%%%%%%%%%%%%%%%%%%%%
%%%%%%%%%%%%%%%%%%%%%%%%%%%%%%%%%%%%%
%%%%%%%%%%                         Max flow model                         %%%
%%%%%%%%%%%%%%%%%%%%%%%%%%%%%%%%%%%%%
\section{Max Flow Model}
\textit{Edge-based formulation}To help us understand the problem, we need to simplify the model, instead of focus on multiple commodity flows, we only focus on one flow with single source and single destination, and ask the basic question: what is the maximum amount of flow along with its processing demand being processed and sent to destination given a graph with a certain node capacity and link capacity for each node and link. Without losing generality and to make the math easier to read, we assume ratio between workload and flow size is 1, i.e. one flow requires one unit of processing. We also assume it is undirected graph as it is closer to real networking setting. %citation needed.

\textbf{Notations:}

B(e): bandwidth capacity for link $e \in E$,

C(v): middlebox processing capacity at node $v \in V$,

$ f(e): $  flow load at edge $e\in E $, 

$ w(e):$ workload demand at edge e, 

$p(v): $   workload processed at node v,

$p(v) = \sum\limits_{in } w(e) - \sum\limits_{out} w(e)  $. 
 
 At the first glance, this problem looks like Max flow problem, however the reduction does not work quite well, and we need to first formulate the generalized Linear Programming to help us understand the underline structure of the problem. 
 \section{Plan}
 In this paper, I plan to:
 \begin{itemize}
 \item{Formulate the linear programming model}
 \item{Design combinatorial algorithm for simple max flow problem}
 \item{Comprehensive analysis for networking design choices}
 \end{itemize}

\bibliographystyle{acm}

\bibliography{references}


\end{document}