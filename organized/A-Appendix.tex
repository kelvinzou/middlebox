\appendix
\section{Appendix}
\subsection{Proof for equivalence between two LPs \autoref{subsec:basicproblem}} \label{subsec:lppaths}
Proof sketch: we first show that we can compose an \emph{edge-based} solution based on a \emph{walk-based} solution and vice versa for a single flow, and then show that we can iteratively place multi-commodity flows. 
\begin{enumerate}
\item show \emph{Direction A:} If there is a \emph{walk-based} LP solution, there is an \emph{edge-based} solution.
\item show \emph{Direction B:} If there is an \emph{edge-based} LP solution, there is a \emph{walk-based} solution using walk decomposition.
\item show the formulations for multi-commodity flows are also equivalent via extending the above approach.
\end{enumerate}

\subsubsection{\emph{walk-based} solution $\rightarrow$ \emph{edge-based} solution}
\begin{proof}
we show that we can easily convert walk-based solution to edge-based solution and all the constraints in edge-based formulation hold.

For each edge e, $f_i(e) =\sum\limits_{\pi\in P: e\in \pi} p_{i,\pi}$.

For each vertex v, $w_i(e) =  \sum\limits_{v'\in \pi, v' \leq v} p_{i,\pi}^v$ ($v'\leq v$ means e' is topologically at or after e on the walk $\pi$). 

Flow conservation holds $ \sum\limits_{(u,v)\in E} f_i(e) $=$ \sum\limits_{\pi\in P, v\in \pi} p_{i, \pi}$ = $\sum\limits_{(v,w )\in E} f_i(e)$. 

Constraints in terms of $B(e), C(v)$ also hold. (A2d,2e)

Relations between $w_i(e), f_i(e)$ also hold: $w_i(e)=  \sum\limits_{v'\in \pi, v' \leq v} p_{i,\pi}^v \leq \sum \limits_{v\in \pi} p_{i,\pi}^v = f_i(e) $, and $w_i(s,v)=f_i(s,v)$ and $w_i(v,t)=0$ are special cases.(A.2f,2g,2h)

\end{proof}

\subsubsection{\emph{edge-based} solution $\rightarrow$ \emph{walk-based} solution}
to prove this; we need to show:
\begin{enumerate}
 \item We can always construct a walk if there is some residual flow left in the graph.
 \item All constraints holds for the updated residual graph.
 
\end{enumerate}

Setup: 
For simplicity, we only construct all walks for a flow each time, so notation wise we can remove $i$. A directed graph $G(V,E)$ with an \emph{edge-based} LP solution, where $f(e)$ is the flow for each edge, $w(e)$ is workload demand at the same edge and $p(v)$ process work done at each vertex v. Build a new graph $G'$: all vertices $V$, and for $\forall e\in E $, if $f(e) >0$, we put a direct edge $e$ in the graph. To help proof, divide a flow into two states, processed and unprocessed $f^1$ and $f^2$; in terms of flow volume $f^1=w$ and $f^2=f-w$. 
\begin{lemma} loops for flows $f^1$ and $f^2$ respectively can be cancelled via flow cancellation without any side effect. 
\end{lemma} 

\begin{proof} it is similar to flow cancellation in a simple graph model: 

(i) for e=(u,v) whereas $min(f^1) >0$, we can simply cancel the unprocessed flow demand by small amount $\epsilon$, and it does not affect the outcome of the flow outside the loop, while we can reduce the flow load and workload demand in the loop without side effect. 

(ii) for e=(u,v) whereas $min(f^2)>0$, we can cancel the processed flow demand by small amount $\epsilon$, and this does not affect the outcome of the flow outside of the loop while we can reduce the flow load in the loop without side effect. 

The intuition behind this is that loop exists due to that some flow needs to borrow some processing capacity from some node(s), so it would ``detour'' a flow in an unprocessed state and get back the flow in a processed state. 
\end{proof}

Introduce an intermediate variable $\rho$ for each edge e where $\rho_{e} = \frac{w(e)}{f(e)} = \frac{ f^1} {f^1+f^2 }$. Run flow loop cancellation for $f^1$ and $f^2$ respectively in $G'$.

\textbf{Note:} after loop cancellation we may still have loops for $f$ as a unitiy. 
\begin{lemma}
$\rho$ has the following property: if there is a cycle for unity flow $f$, there is always at least one edge with $\rho =1$ and one edge with $\rho =0$.
\end{lemma}
\begin{proof}
 This can be easily inferred from Lemma A.1.
\end{proof}



\begin{algorithm}[ht]
  \SetAlgoLined\DontPrintSemicolon
   \KwData{$G'(V, E)$, $ w(e)$, $f(e)$ for $\forall e \in E$ and $p(v)$ for $\forall v \in V$ }
\KwResult{$f(\pi)$, $p(\pi,v)$ where $v\in\pi$ }
\SetKwFunction{walk}{Walk Construction} \label{walk construction}
  \SetKwFunction{flow}{Flow Placement} \label{flow placement}
  \SetKwProg{myalg}{Algorithm}{}{}
  \myalg{\walk{}}  {
\emph{//Construct walk from $s\rightarrow v$ and $v\rightarrow t$}\;
From v run backward traversal, pick an incoming directed edge with $ max( \rho_{in} )  $ where $\rho_{in} \equiv \frac{ w(e_{in})}{f(e_{in})}$\;
From v run forward traversal, pick an outgoing directed edge with $ min(\rho_{out} ) $ where $\rho_{out} \equiv \frac{ w(e_{out})}{f(e_{out})} $\;
 \KwRet $\pi$\;}{}
%   \nl \KwRet $\pi$\;}{}
  \setcounter{AlgoLine}{0}
  \SetKwProg{myproc}{Algorithm}{}{}
  \myproc{\flow{}}{
  \BlankLine
\While{$\exists v; p(v) >0$}
{
 \emph{//walk representation $\pi\equiv <v_1, \dots, v_k> \equiv <e_1, \dots, e_{k-1}>$}\;
 $\pi$ = \walk{}\;
$p_\pi = \min\{f^1(e^a), f^2(e^b), p(v)\},\; e^a\in <e_1,\dots, u\rightarrow v>, e^b\in <v\rightarrow w,\dots,e_{k-1}>$\;
	$p_\pi^v=p_\pi$\;
 	\For{$u\in \pi$ and $u\not=v$ }
	{
	$p_\pi^v=0$\;
	}
	$C(v) = C(v) - p_\pi$\;
	$p(v) = p(v) - p_\pi$\;
	\For{$i \leftarrow 1 $ \KwTo $k-1$}{
	$f(e_i)=f(e_i)-p_\pi$\;
	$B(e_i)=B(e_i)-p_\pi$\;
	}
}
}
\caption{Walk Decomposition}
\end{algorithm}


\begin{lemma}  [Walk Construction] Algorithm 1 can always generate a walk with non-zero flow from source to sink if there exists any $v$ where $p(v) >0$, and the algorithm can converge at $O(V^2)$ \end{lemma}
\begin{proof}
\textit{First}, from Lemma A.2, the walk cannot loop a cycle twice from [Walk Construction]. Since downstream traversal keeps picking $\min \rho$ while upstreaming traversal keeps picking $\max \rho$, so we never pick the same edge twice. Since $p(v)>0$ so at the same node there must be one upstream edge with $\rho>0$ and downstream edge with $\rho<1$. Since the same edge is never picked twice so there is no loop in terms of $f^1$ and $f^2$. The walk consists of two DAGs, one is from source to $v$ and one is from $v$ to sink, the walk is a DAG as well.

\textit{Second} we need to show for a certain walk $\pi; p_\pi>0$. Since $p_\pi =  \min\{f^1(e^a), f^2(e^b), p(v)\}$; at node v where $p(v)>0$, so we have $f^1(e_{in})>0$ and $f^2(e_{out})>0$ at vertex $v$. Since we only pick $\max\{\rho\}$ for upstream traversal, so for $\forall e^a; f^1(e^a)>0$. The same reason we have $\forall e^b; f^2(e^b)>0$.

For a single flow, after each iteration, we either take out one edge or one vertex, and the runtime for each iteration is $O(|V|)$ for traversal. As we iterate through $O(|V|)$ vertices and $O(|E|)$ for edges so the runtime total will be $O(|E|+|V|)\cdot |V|) = O(|V|\cdot |E|)$.
\end{proof}


\begin{lemma} [Flow Placement] Algorithm 1 conserves all the constraints for the reduced graph.
\end{lemma}
\begin{proof}
we show that all the constraints are satisfied:

for A.2b:
$\forall v \in \pi; \sum\limits_{in}  f(e) - \sum\limits_{out} f(e)=\sum\limits_{in \not=e_i}  f(e) - \sum\limits_{out\not=e_{i+1} } f(e) +[f(e_i)-p_\pi ] - [f(e_{i+1}) p_\pi] = 0 $

A.2d:
$\forall e \in \pi; f(e) = f(e)-p_\pi \leq B(e)-p_\pi=B^{new}(e)$ 

A.2e:
$\forall v \in \pi; p(v) - p_\pi \leq C(v) - p_\pi =C^{new}(v)$

A.2f and A.2g are ensured by the algorithm, since $v\not=s $ and $ v\not=t$.

A.2h constraints are satisfied by numerical relations.
\end{proof}

\subsubsection{Multi-Commodity Flow}
For MCF, we can use the same approach above. For a graph with K source-sink paired flows,  we iterate $i = 1 \dots K$, for each flow we genrate a $G'$ and exhaustively decompose walks for $f_i$ and it is easy to see that all the constraints still hold after flow $i$ has been removed.  
In particular, we have :
A.2b, A.2c, A.2f and A.2g  hold for all the flows left after one flow is removed;
A.2d: $\forall i,\forall e;  \sum\limits_{l=i}^K f_l(e) - f_i\leq B(e) - f_i(e)$;
A.2e: $\forall i,\forall v;  \sum\limits_{l=i}^K p_l(v) -p_i(v)\leq C(v) - p_i(v)$.

\subsection{Proof for formulation about flow size change in \autoref{subsec:basicproblem}} \label{subsec:lppaths}
\begin{proof}
The algorithm of constructing walks for edge-based LP w.r.t. flow size change is very similar to the approach above. However we need to ``restore'' the processed flow to ``raw'' flow size before constructing the walks. In particular, using the $f^1, f^2$ notation to represent pre and post processed flows, we restore $f^2$ to $\hat{f}^2$ with the multiplicative factor $r$, $ \forall e, \hat{f}^2(e)=f^2(e)/r$. Everything is the same as previous algorithm after this step. Once we get $f^1$ and $\hat{f}^2$, we simply convert back to $f^2$ via $f^2= \hat{f}^2 \cdot r$.
\end{proof}